\chapter{Seznámení}\label{sec:Introduction}
V dnešní společnosti je překlad mezi různými jazyky velmi vyhledávaným a klíčovým nástrojem pro komunikaci a sdílení informací.
S rychlým pokrokem technologií stále více uživatelů vyhledává strojového překladu a využívají automatické systémy pro překlad textu.
Technologie v oblasti strojového překladu se stávají stále běžnějšími, dostupnějšími pro širší škálu uživatelů.
Tyto systémy pro překlad, založené na algoritmech strojového učení, jsou schopny generovat překlady s překvapivou úrovní přesnosti.
I když se technologie strojového překladu neustále zdokonalují, nastává problém s rozpoznáním a detekcí strojově přeložených textů a přirozeného lidského překladu.
Jsou situace ve kterých je důležité rozlišit, zda se jedná o lidský přirozený překlad nebo strojově generovaný.

Cílem tohoto projektu je vytvořit efektivní a spolehlivý model pro detekci strojově přeložených textů pomocí technik strojového učení.
Práce zkoumá porovnání mezi přirozeným lidským překladem a strojově psanými textovými soubory, které se použijí jako vstupní data pro detekční model.
Následně se budou vyvíjet a trénovat algoritmy strojového učení, které budou schopny rozpoznat specifické znaky strojově přeloženého textu a odlišit je od lidského překladu.
Výsledkem tohoto projektu je vytvořen klasifikátor, jakož detekční model, který je schopen s přijatelnou přesností identifikovat strojově přeložené texty nad přirozeným lidským překladem.

První kapitola se věnuje problematice zpracování přirozeného jazyka (NLP).
Zabývá se konkrétně přirozeně lidskou sentimentální analýzou a popisuje následné techniky pro převod textu do strojově srozumitelného formátu.
V první části je podrobněji vysvětlena sentimentální analýza.
Následně se kapitola zaměřuje na techniky převodu textu do strojově srozumitelného formátu.
Těmito technikami se rozumí proces transformace přirozeného jazyka na formu, kterou může snadno zpracovat počítač či jiný stroj.
Tato kapitola poskytuje ucelený přehled sentimentální analýzy a technik převodu textu do strojově srozumitelného formátu, které jsou využívány v oblasti zpracování přirozeného jazyka.

Druhá kapitola stručně popisuje koncept neuronových sítí. Dále se věnuje modifikacím neuronových sítí, které jsou známy jako rekurentní neuronové sítě (RNN), a jejich vylepšením.
Tato druhá kapitola poskytuje základní přehled o neuronových sítích a jejich využití v kontextu rekurentních neuronových sítí a jejich vylepšení.

V třetí kapitole je popsána jedna z nejmodernějších forem neuronových sítí -~síť~Transformers.
Dále zdůrazňuje její význam v oblasti zpracování přirozeného jazyka a dalších úloh spojených se sekvencemi dat.

Předposlední kapitola poskytuje detailní přehled o výsledcích praktické části, kde byly zkoumány různé architektury modelů a jejich konfigurace. Tyto výsledky jsou klíčové vyhodnocení úspěšnosti a vhodnosti jednotlivých modelů pro daný problém.

V poslední kapitole je poskytnuto shrnutí dosažených výsledků natrénovaných modelů a navrhnuty možnosti jejich zlepšení.

\endinput