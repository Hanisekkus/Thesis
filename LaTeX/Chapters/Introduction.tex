\chapter{Seznámení}\label{sec:Introduction}
V dnešní společnosti je překlad mezi různými jazyky klíčovým nástrojem pro komunikaci a sdílení informací.
S rychlým pokrokem technologií v oblasti strojového překladu se stává stále běžnějším používat automatické systémy pro překlad textu.
Tyto systémy, založené na algoritmech strojového učení, jsou schopny generovat překlady s překvapivou úrovní přesnosti.
Nicméně, přestože se technologie strojového překladu neustále zdokonalují, stále existuje problém s detekcí strojově přeložených textů.
V některých případech může být důležité rozlišovat mezi lidským překladem a strojovým překladem.
Cílem tohoto projektu je vytvořit efektivní a spolehlivý model pro detekci strojově přeložených textů pomocí technik strojového učení.
Práce se zabývá detekcí nad lidsky a strojově psanými textovými soubory, které se použijí jako vstupní data pro detekční model.
Následně se budou vyvíjet a trénovat algoritmy strojového učení, které budou schopny rozpoznat specifické znaky strojově přeloženého textu a odlišit je od lidského překladu.
Výsledkem tohoto projektu bude vytvoření detekčního modelu, který bude schopen s přijatelnou přesností identifikovat strojově přeložené texty.

První kapitola se věnuje problematice zpracování přirozeného jazyka (NLP). Zabývá se konkrétně sentimentální analýzou a popisuje následné techniky pro převod textu do strojově srozumitelného formátu.
V první části je podrobněji vysvětlena sentimentální analýza. Následně se kapitola zaměřuje na techniky převodu textu do strojově srozumitelného formátu. Těmito technikami se rozumí proces transformace přirozeného jazyka na formu, kterou může snadno zpracovat počítač či jiný stroj.
Tato kapitola poskytuje ucelený přehled sentimentální analýzy a technik převodu textu do strojově srozumitelného formátu, které jsou využívány v oblasti zpracování přirozeného jazyka.

Druhá kapitola stručně popisuje koncept neuronových sítí. Dále se věnuje modifikacím neuronových sítí, které jsou známy jako rekurentní neuronové sítě (RNN), a jejich vylepšením.
Tato druhá kapitola poskytuje základní přehled o neuronových sítích a jejich využití v kontextu rekurentních neuronových sítí a jejich vylepšení.

V třetí kapitole je popsána jedna z nejmodernějších forem neuronových sítí -~síť~Transformers.
Dále zdůrazňuje její význam v oblasti zpracování přirozeného jazyka a dalších úloh spojených se sekvencemi dat.

Předposlední kapitola poskytuje detailní přehled o výsledcích praktické části, kde byly zkoumány různé architektury modelů a jejich konfigurace. Tyto výsledky jsou klíčové vyhodnocení úspěšnosti a vhodnosti jednotlivých modelů pro daný problém.

V poslední kapitole je poskytnuto shrnutí dosažených výsledků natrénovaných modelů a navrhnuty možnosti jejich zlepšení.

\endinput