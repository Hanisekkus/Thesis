\chapter{Závěr diplomové práce}
V této diplomové práci jsem se snažil vytvořit strojově naučený model, který by byl schopný detekovat strojově přeložené texty.
Provedl jsem řadu experimentů s různými architekturami a úpravami, abych našel optimální modely, které by s využitím znalostí z oblasti zpracování přirozeného jazyka a sentimentální analýzy dokázaly rozlišovat lidsky psaný a strojově generovaný text.
Navzdory nízké úspěšnosti považuji modely za potenciálně schopné řešit tento problém.

Hlavní překážkou byl nedostatek dostupných dat.
Vytvoření dostatečně velkého datasetu představovalo výzvu vzhledem k potřebě shromáždit rozsáhlý textový korpus a provést jeho překlad pomocí nástrojů, které nejsou primárně určeny pro překlad velkých textových segmentů, ale spíše se zaměřují na překlad jednotlivých vět nebo odstavců.
Tento proces by vyžadoval značné zdroje a prostředky.
Bohužel, z důvodu těchto omezení nebylo možné vytvořit dostatečný dataset pro účely této studie.

\section{Možná vylepšení}
Jednou z hlavních možností vylepšení je rozšíření textového datového souboru, a to významným způsobem.
Důkladné rozšíření datového souboru by mohlo přispět k zlepšení přesnosti modelu.
Dále na základě rozšířeného datového souboru by bylo vhodné průběžně upravovat parametry architektur s cílem dosáhnout lepších výsledků.
Pravidelné úpravy parametrů a experimentování s různými nastaveními mohou vést ke zdokonalení modelů.

\section{Zkušenost}
Díky této práci jsem získal neuvěřitelné množství znalostí, a to od samého začátku.
Zjistil jsem, jak je důležité mít k dispozici skvělý dataset, ze kterého lze vypočítat požadované výsledky.
Dále jsem získal zkušenost s vytvářením vlastních modelů a také s technikou fine-tuningu před-trénovaných modelů.

Nabyté zkušenosti mi umožnily porovnávat různé modely a hodnotit je na základě specifických kritérií.
Dále jsem získal schopnost detekovat nedostatky a problémy v modelech, pokud se objeví.
Všechny tyto dovednosti a znalosti jsem získal prostřednictvím této práce.
\endinput
